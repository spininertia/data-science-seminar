\documentclass{article} % For LaTeX2e
\usepackage{nips13submit_e,times}
\usepackage{hyperref}
\usepackage{url}
%\documentstyle[nips13submit_09,times,art10]{article} % For LaTeX 2.09


\title{Literature Survey of Topic Model}


\author{
Wei Chen \\
Department of Computer Science\\
Carnegie Mellon University\\
Pittsburgh, PA 15213 \\
\texttt{weichen1@andrew.cs.edu} \\
\And
Siping Ji\\
Department of Computer Science\\
Carnegie Mellon University\\
Pittsburgh, PA 15213 \\
\texttt{sipingji@cmu.edu} \\
\AND
Da Teng \\
Department of Computer Science\\
Carnegie Mellon University\\
Pittsburgh, PA 15213 \\
\texttt{dateng@cmu.edu} \\
}

% The \author macro works with any number of authors. There are two commands
% used to separate the names and addresses of multiple authors: \And and \AND.
%
% Using \And between authors leaves it to \LaTeX{} to determine where to break
% the lines. Using \AND forces a linebreak at that point. So, if \LaTeX{}
% puts 3 of 4 authors names on the first line, and the last on the second
% line, try using \AND instead of \And before the third author name.

\newcommand{\fix}{\marginpar{FIX}}
\newcommand{\new}{\marginpar{NEW}}

\nipsfinalcopy % Uncomment for camera-ready version

\begin{document}


\maketitle

\begin{abstract}
Study most important techniques invented during the development of Topic Model.
\end{abstract}

\section{Introduction}
    \label{sec:intro}
    Topic Model has been a successful tool for managing large text data set in different fields. For example, it's been applied to index latent \emph{term} in Information Retrieval\cite{deerwester1990indexing}, which proves its power to mining hidden states. Different models and algorithms are have been proposed during its development, for example, \cite{dumais1995latent}, \cite{blei2003latent}and\cite{hofmann1999probabilistic}. The report will focus that representative results to give a detailed overview of this field. 
    
\section{Submission of papers to NIPS 2013}

NIPS requires electronic submissions.  The electronic submission site is  
\begin{center}
   \url{http://papers.nips.cc}
\end{center}

\subsection{Retrieval of style files}

The style files for NIPS and other conference information are available on the World Wide Web at
\begin{center}
   \url{http://www.nips.cc/}
\end{center}

\verb+nips11submit_09.sty+ (to be used with \LaTeX{} version 2.09) and
\verb+nips11submit_e.sty+ (to be used with \LaTeX{}2e). The file
\verb+nips2013.tex+ may be used as a ``shell'' for writing your paper. All you
have to do is replace the author, title, abstract, and text of the paper with
your own. The file
\verb+nips2013.rtf+ is provided as a shell for MS Word users.

\section{General formatting instructions}

\section{Citations, figures, tables, references}
\label{others}

These instructions apply to everyone, regardless of the formatter being used.

\subsection{Citations within the text}

Citations within the text should be numbered consecutively. The corresponding
number is to appear enclosed in square brackets, such as [1] or [2]-[5]. The
corresponding references are to be listed in the same order at the end of the
paper, in the \textbf{References} section. (Note: the standard
\textsc{Bib\TeX} style \texttt{unsrt} produces this.) As to the format of the
references themselves, any style is acceptable as long as it is used
consistently.


\subsection{Figures}

\begin{figure}[h]
\begin{center}
%\framebox[4.0in]{$\;$}
\fbox{\rule[-.5cm]{0cm}{4cm} \rule[-.5cm]{4cm}{0cm}}
\end{center}
\caption{Sample figure caption.}
\end{figure}

\subsection{Tables}

All tables must be centered, neat, clean and legible. Do not use hand-drawn
tables. The table number and title always appear before the table. See
Table~\ref{sample-table}.

Place one line space before the table title, one line space after the table
title, and one line space after the table. The table title must be lower case
(except for first word and proper nouns); tables are numbered consecutively.

\begin{table}[t]
\caption{Sample table title}
\label{sample-table}
\begin{center}
\begin{tabular}{ll}
\multicolumn{1}{c}{\bf PART}  &\multicolumn{1}{c}{\bf DESCRIPTION}
\\ \hline \\
Dendrite         &Input terminal \\
Axon             &Output terminal \\
Soma             &Cell body (contains cell nucleus) \\
\end{tabular}
\end{center}
\end{table}


\section{Preparing PostScript or PDF files}

\subsection{Margins in LaTeX}
 
Most of the margin problems come from figures positioned by hand using
\verb+\special+ or other commands. We suggest using the command
\verb+\includegraphics+
from the graphicx package. Always specify the figure width as a multiple of
the line width as in the example below using .eps graphics
\begin{verbatim}
   \usepackage[dvips]{graphicx} ... 
   \includegraphics[width=0.8\linewidth]{myfile.eps} 
\end{verbatim}
or % Apr 2009 addition
\begin{verbatim}
   \usepackage[pdftex]{graphicx} ... 
   \includegraphics[width=0.8\linewidth]{myfile.pdf} 
\end{verbatim}
for .pdf graphics. 
See section 4.4 in the graphics bundle documentation (\url{http://www.ctan.org/tex-archive/macros/latex/required/graphics/grfguide.ps}) 
 
A number of width problems arise when LaTeX cannot properly hyphenate a
line. Please give LaTeX hyphenation hints using the \verb+\-+ command.


\subsubsection*{Acknowledgments}
Test citation. \cite{hofmann1999probabilistic}

\bibliography{BIB/siping_ref,BIB/weichen_ref,BIB/dateng_ref}
\bibliographystyle{plain}

\end{document}
